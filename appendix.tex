\chapter{Symmetry breaking for Abelian Gauge Theory}
In this chapter, we briefly explain the behavior of gauge theories under symmetry breaking. \\


So, let us first consider a Lagrangian with the form:

\begin{equation}
    \mathcal{L} = \partial_{\mu}\phi^{\dagger}\partial^{\mu}\phi - \mu^2\left(\phi^{\dagger} \phi\right) - \lambda \left(\phi^ {\dagger}\phi\right)^2
\end{equation}
So there is gauge symmetry, but there's not a gauge field.\\
If $\mu^2<0$ the field has its minimum in:
\begin{equation}
    \langle \phi \rangle_0 \equiv v = \sqrt{\frac{\mu^2}{\lambda}}
\end{equation}



We can parametrize $\phi$ with a field that describes the fluctuations around the minimum:
\begin{equation}
    \phi(x) = \left(v + \eta(x)\right)e^{i \theta(x)}
\end{equation}
where $\theta(x)$ takes into account the gauge symmetry.\\
Substituting into the Lagrangian:
\begin{equation}
    \mathcal{L}=\partial_{\mu}\eta\partial^{\mu}\eta - 4 \lambda v^2 \eta^2 + v^2 \partial_{\mu}\theta\partial^{\mu}\theta + O(\eta^3)
\end{equation}
So, like in the global case, we find two particles, one massive, $eta$, and one massless, $\theta$, that is associated with the gauge symmetry, so it is a gauge boson.\\
Let us now consider a Lagrangian with the form:
\begin{equation}
    \mathcal{L} = \left(D_{\mu}\phi^{\dagger}\right) \left(D^{\mu} \phi\right) - \mu^2\left(\phi^{\dagger} \phi\right) - \lambda \left(\phi^ {\dagger}\phi\right)^2 - \frac{1}{4} F_{\mu\nu}F^{\mu\nu},
\end{equation}
where $D_{\mu}= \partial_{\mu} + i e A_{\mu}$ is the covariant derivative of the interaction field and keeps the field invariant under gauge transformations.
This Lagrangian is invariant under any local transformation from the $U(1)$ group.\\
If we consider the case $\lambda >0, \mu^2<0$ we obtain:
\begin{equation}
    \begin{cases}
        \langle\phi\rangle_0 \equiv \frac{v}{\sqrt{2}} = \sqrt{\frac{\mu^2}{2\lambda}}\\
        \langle A_{\mu} \rangle_0 = 0
    \end{cases}
\end{equation}
We can parametrize the field in this case as well:
\begin{equation}
    \phi(x) = \left(v + \eta(x)\right)e^{i \theta(x)}
\end{equation}
If we now consider the gauge transformations both for the field and the interaction field:
\begin{equation}
    \begin{cases}
        \phi(x)\rightarrow e^{-i\Lambda(x)}\\
        A_{\mu}(x)\rightarrow A_{\mu}(x) - \frac{1}{\alpha}\partial_{\mu}\Lambda(x)
    \end{cases}
\end{equation}
and we fix the gauge, that is, we decide the ambiguity on the gauge transformation that leaves the system invariant, by choosing for example:
\begin{equation*}
    \Lambda(x) = \theta(x)
\end{equation*}

we find that the reparametrized field now has a simpler form:
\begin{equation}
    \phi(x) = \eta(x) + v
\end{equation}

Inserting in the Lagrangian we find:
\begin{equation}
      \mathcal{L}= - \frac{1}{4} F_{\mu\nu}F^{\mu\nu} + \alpha^2 v^2A_{\mu}A^{\mu} + \partial_{\mu}\eta\partial^{\mu}\eta - 4 \lambda v^2 \eta^2 + O(\eta^3)
\end{equation}

So, we can see that, if the Lagrangian has a gauge symmetry, and we add a mechanism of spontaneous symmetry breaking, the Lagrangian keeps its gauge symmetry (we can still apply the transformations previously defined without changing the Lagrangian), but we also obtain a new particle, $\eta$, massive, and the gauge field as well obtains mass.\\
This mechanism is called Abelian Higgs mechanism.\\
The physical case uses a non-Abelian group, and in this way it theorizes four gauge bosons.

\chapter{Summary of the properties of elementary particles}
\label{app_particles}
A summary of the most important characteristics of SM elementary particles is reported.
\section{Leptonic sector}
\begin{table}[hb]
    \centering
    \begin{tabular}{c|c|c|c}
         Particle & mass [MeV]& Q & Y \\ \hline
         $e$ & 0.511 & -1  & -2\\ \hline
         $\nu_e$& $<0.8 10^{-6}$& 0 & 0 \\\hline    
         $\mu$ & 105.6& -1 &-2 \\\hline
         $\nu_{\mu}$ & & 0  & 0\\\hline
         $\tau$ & 1776.86 & -1 & -2\\\hline
         $\nu_{\tau}$ & & 0 & 0\\\hline
         
        
    \end{tabular}
    \caption{Summary of the leptonic sector}
    \label{lepton_summary}
\end{table}
\section{Quark sector}
\begin{table}[hb]
    \centering
    \begin{tabular}{c|c|c|c}
         Particle & mass [MeV] & Q &  Y \\ \hline
         u & 0.511 & +2/3  & \\ \hline
         d & & -1/3 & \\\hline
         c & 0.511 & +2/3  & \\ \hline
         s & & -1/3 & \\\hline         
         t & 0.511 & +2/3  & \\ \hline
         b & & -1/3 & \\\hline
         
 
    \end{tabular}
    \caption{Summary of the quark sector}
    \label{quark_summary}
\end{table}
\section{Gauge bosons sector}
\begin{table}[hb]
    \centering
    \begin{tabular}{c|c|c|c|c}
         Particle & mass [GeV] & S & Q  & Y \\ \hline
         W & 80.3 &1& $\pm$1 & 0 \\ \hline
         Z & 91.19 & 1 & 0 & 0 \\\hline
         $\gamma$ & 0&1 & 0 &  0\\\hline
         $g$ & 0 & 1& 0 & 0 \\\hline
         H & 125.25 & 0 & 0 & 1 \\\hline 
 
    \end{tabular}
    \caption{Summary of the gauge boson sector}
    \label{gauge_summary}
\end{table}

\chapter{$\kappa$-framework and results}
\label{kappa_framework}
The $\kappa$-framework is one of the tests on the SM. It consists of letting the various coupling constants vary and then studying, using a maximum likelihood test, the most probable value for the coupling constant.\\
In particular, every coupling constant is multiplied by a scaling factor $\kappa$